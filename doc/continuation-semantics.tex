% -*- coding: utf-8; -*-
% vim: set fileencoding=utf-8 :
\documentclass[english, references=cleveref]{programming}
%% First parameter: the language is 'english'.
%% Second parameter: use 'submission' for initial submission, remove it for camera-ready (see 5.1)

\usepackage[backend=biber]{biblatex}
\addbibresource{biblio.bib}


%
% Packages and Commands specific to article (see 3)
%
% These ones  are used in the guide, replace with your own.
% 
\usepackage{multicol}
\usepackage{mathpartir}
\usepackage{nicefrac}
\usepackage{subcaption}
\usepackage{amsthm}

%% Syntax
\newcommand{\synvar}[1]{\ensuremath{#1}}
\newcommand{\synkeyword}[1]{\textcolor{red!60!black}{\texttt{#1}}}
\newcommand{\synpunct}[1]{\textcolor{black!40!white}{\texttt{#1}}}
\newcommand{\synname}[1]{\ensuremath{\mathsf{#1}}}
\newcommand{\synbool}{\synkeyword{bool}}
\newcommand{\synnum}{\synkeyword{num}}
\newcommand{\syndate}{\synkeyword{date}}
\newcommand{\synvec}{\synkeyword{vec~}}
\newcommand{\synopt}{\synkeyword{opt~}}
\newcommand{\synrule}{\synkeyword{rule~}}
\newcommand{\synlet}{\synkeyword{let~}}
\newcommand{\synletstar}{\synkeyword{let}\synpunct{*}~}
\newcommand{\synin}{\synkeyword{~in~}}
\newcommand{\synis}{\synkeyword{~is~}}
\newcommand{\synif}{\synkeyword{if~}}
\newcommand{\synthen}{\synkeyword{~then~}}
\newcommand{\synelse}{\synkeyword{~else~}}
\newcommand{\synelsenewline}{\synkeyword{else~}}
\newcommand{\syncall}{\synkeyword{call~}}
\newcommand{\synscope}{\synkeyword{scope~}}
\newcommand{\synequal}{\synpunct{~=~}}
\newcommand{\synjust}{~\synpunct{:\raisebox{-0.9pt}{-}}~}
\newcommand{\syntyped}{~\synpunct{:}~}
\newcommand{\syncomma}{\synpunct{,}}
\newcommand{\syndot}{\synpunct{.}~}
\newcommand{\synunit}{\synpunct{()}}
\newcommand{\synunitt}{\synkeyword{unit}}
\newcommand{\syntrue}{\synkeyword{true}}
\newcommand{\synfalse}{\synkeyword{false}}
\newcommand{\synforall}{\synpunct{$\forall$}}
\newcommand{\synop}{\synpunct{\odot}}
\newcommand{\synlambda}{\synpunct{$\lambda$}~}
\newcommand{\synand}{~\synkeyword{and}~}
\newcommand{\synor}{~\synkeyword{or}~}
\newcommand{\synnot}{\synkeyword{not}~}
\newcommand{\synimply}{~\synkeyword{imply}~}
\newcommand{\synlparen}{\synpunct{(}}
\newcommand{\synrparen}{\synpunct{)}}
\newcommand{\synlsquare}{\synpunct{[}}
\newcommand{\synrsquare}{\synpunct{]}}
\newcommand{\synlbracket}{\synpunct{\{}}
\newcommand{\synrbracket}{\synpunct{\}}}
\newcommand{\synlangle}{\synpunct{$\langle$}}
\newcommand{\synlanglemid}{\synpunct{$\langle\!|$}~}
\newcommand{\synrangle}{\synpunct{$\rangle$}}
\newcommand{\synmid}{\synpunct{~$|$~}}
\newcommand{\synemptydefault}{\synvar{\varnothing}}
\newcommand{\synerror}{\synvar{\circledast}}
\newcommand{\synempty}{\synemptydefault}
\newcommand{\synconflict}{\synerror}
\newcommand{\synstar}{\synpunct{~$*$~}}
\newcommand{\synvardef}{\synkeyword{definition~}}
\newcommand{\synscopecall}{\synkeyword{scope\_call~}}
\newcommand{\synlarrow}{~\synpunct{$\leftarrow$}~}
\newcommand{\synarrow}{~\synpunct{$\rightarrow$}~}
\newcommand{\synArrow}{~\synpunct{$\Rightarrow$}~}
\newcommand{\synellipsis}{\synpunct{,$\ldots$,}}
\newcommand{\synlistellipsis}{\synpunct{;$\ldots$;}}
\newcommand{\syndef}{$ ::= $}
\newcommand{\synalt}{\;$|$\;}
\newcommand{\synhole}{\synvar{\cdot}}
\newcommand{\syncrashifempty}{\synkeyword{crash\_if\_empty}}
\newcommand{\synnone}{\texttt{None}}
\newcommand{\synsome}{\texttt{Some}~}
\newcommand{\synmatch}{\synkeyword{match}~}
\newcommand{\synwith}{~\synkeyword{with}~}
\newcommand{\synoption}{\texttt{option}\;}
\newcommand{\synraise}{\synkeyword{raise}\;}
\newcommand{\synemptyerror}{\texttt{EmptyError}}
\newcommand{\synconflicterror}{\texttt{ConflictError}}
\newcommand{\syntry}{\synkeyword{try}\;}
\newcommand{\synlist}{\;\texttt{list}}
\newcommand{\syncons}{\synpunct{::}}
\newcommand{\fstar}{F$^\star$\xspace}

\newcommand{\ghostvvalue}{}
\newcommand{\ghostvalue}{}
\newcommand{\ghostbool}{}
\newcommand{\ghostint}{}

%% Typing
\newcommand{\typctx}[1]{\textcolor{orange!90!black}{\ensuremath{#1}}}
\newcommand{\typempty}{\typctx{\varnothing}}
\newcommand{\typcomma}{\typctx{,\;}}
\newcommand{\typvdash}{\typctx{\;\vdash\;}}
\newcommand{\typcolon}{\typctx{\;:\;}}
\newcommand{\typlpar}{\typctx{(}}
\newcommand{\typrpar}{\typctx{)}}


%% Evaluation
\newcommand{\exctx}[1]{\textcolor{blue!80!black}{\ensuremath{#1}}}
\newcommand{\exeemptysubdefaults}{\exctx{\mathsf{empty\_count}}}
\newcommand{\execonflictsubdefaults}{\exctx{\mathsf{conflict\_count}}}
\newcommand{\Omegaarg}{\Omega_{arg}}
\newcommand{\excaller}{\exctx{\complement}}
\newcommand{\excomma}{\exctx{,}\;}
\newcommand{\exvdash}{\;\exctx{\vdash}\;}
\newcommand{\exempty}{\exctx{\varnothing}}
\newcommand{\exemptyv}{\exctx{\varnothing_v}}
\newcommand{\exemptyarg}{\exctx{\varnothing_{arg}}}
\newcommand{\exvarmap}{\exctx{~\mapsto~}}
\newcommand{\exscopemap}{\exctx{~\rightarrowtail~}}
\newcommand{\exArrow}{\exctx{~\Rrightarrow~}}
\newcommand{\exeq}{\exctx{\;=\;}}
\newcommand{\exeval}{\exctx{\;\longrightarrow\;}}
\newcommand{\exevalstar}{\exctx{\;\longrightarrow^*\;}}
\newcommand{\exat}{\exctx{\texttt{\;@\;}}}
\newcommand{\exsemicolon}{\exctx{;~}}
\newcommand{\excomp}{\dashrightarrow}



% Scopelang

%% Reduction of the scope language
\newcommand{\redctx}[1]{\textcolor{green!50!black}{\ensuremath{#1}}}
\newcommand{\reduces}{\redctx{~\rightsquigarrow~}}
\newcommand{\redvdash}{\redctx{\;\vdash\;}}
\newcommand{\redturnstile}[1]{\;\ensuremath{\redctx{\vdash}_{#1}}\;\;}
\newcommand{\redcomma}{\redctx{,\;}}
\newcommand{\redsc}{\redctx{;\;}}
\newcommand{\redcolon}{\redctx{\;:\;}}
\newcommand{\redempty}{\redctx{\varnothing}}
\newcommand{\redproduce}{\;\redctx{\Rrightarrow}\;}
\newcommand{\redellipsis}{\redctx{,\ldots,~}}

\newcommand{\redlparen}{\redctx{(}}
\newcommand{\redrparen}{\redctx{)}}
\newcommand{\redequal}{\redctx{~=~}}
\newcommand{\redinit}{\redctx{\mathsf{init\_subvars}}}


% Dcalc

%% Reduction of the defaults
\newcommand{\compctx}[1]{\textcolor{yellow!70!black}{\ensuremath{#1}}}
\newcommand{\compkeyword}[1]{\textcolor{yellow!60!black}{\texttt{#1}}}
\newcommand{\compiles}{\ensuremath{~\compctx{\rightrightarrows}~}}
\newcommand{\compnormal}{\compkeyword{normal}}
\newcommand{\compdefault}{\compkeyword{default}}
\newcommand{\compcons}{\compkeyword{cons}}
\newcommand{\compvdash}{\compctx{\;\vdash\;}}
\newcommand{\compok}{\;\;\compkeyword{ok}}



%% Big step semantics
\newcommand{\bigctx}[1]{\textcolor{teal!70!black}{\ensuremath{#1}}}
\newcommand{\bigkeyword}[1]{\textcolor{teal!60!black}{\texttt{#1}}}
\newcommand{\bigstep}{\ensuremath{~\bigctx{\Downarrow}~}}
\newcommand{\bigsep}{\ensuremath{~\bigctx{\vdash}~}}

\newcommand{\bignil}{\synpunct{[]}}
\newcommand{\bigcons}{\synpunct{:\!:}}



\newcommand{\transctx}[1]{\textcolor{pink!70!black}{\ensuremath{#1}}}
\newcommand{\trans}{\ensuremath{\transctx{\leadsto}}}

%% Generation of verification conditions
\newcommand{\vckeyword}[1]{\textcolor{green!60!black}{\texttt{#1}}}
\newcommand{\vcctx}[1]{\textcolor{green!60!black}{\ensuremath{#1}}}
\newcommand{\vcnoempty}{\vckeyword{no\_empty}}
\newcommand{\vcnoconflict}{\vckeyword{no\_conflict}}
\newcommand{\vcgeneratesnoempty}{\vcctx{\;\Rightarrow_\synemptydefault\;}}
\newcommand{\vcgeneratesnoconflict}{\vcctx{\;\Rightarrow_\synerror}\;}
\newcommand{\vcsep}{\ensuremath{~\vcctx{\vdash}~}}

\newcommand{\vcnil}{\synpunct{[]}}
\newcommand{\vccons}{\synpunct{,}}

%% Continuations
\newcommand{\synClo}{\synkeyword{\texttt{Clo}}}
\newcommand{\synCClo}{\synkeyword{\texttt{CClo}}}
\newcommand{\synDef}{\synkeyword{\texttt{Def}}}
\newcommand{\synReturn}{\synkeyword{\texttt{Return}}}



\newtheorem{theorem}{Theorem}
\newtheorem{lemma}[theorem]{Lemma}


\newcommand{\leval}{\left\langle\!\!\left\langle}
\newcommand{\reval}{\right\rangle\!\!\right\rangle}
\newcommand{\lcont}{\left[\!\!\left[}
\newcommand{\rcont}{\right]\!\!\right]}
\newcommand{\cons}{:\!:}



%%


%%%%%%%%%%%%%%%%%%
%% These data MUST be filled for your submission. (see 5.3)
\paperdetails{
  %% perspective options are: art, sciencetheoretical, scienceempirical, engineering.
  %% Choose exactly the one that best describes this work. (see 2.1)
  perspective=theoretical,
  %% State one or more areas, separated by a comma. (see 2.2)
  %% Please see list of areas in http://programming-journal.org/cfp/
  %% The list is open-ended, so use other areas if yours is/are not listed.
  area={Program verification, {Interpreters, virtual machines and compilers}, Domain-Specific programming},
  %% You may choose the license for your paper (see 3.)
  %% License options include: cc-by (default), cc-by-nc
  % license=cc-by,
}
%%%%%%%%%%%%%%%%%%

%%%%%%%%%%%%%%%%%%
%% These data are provided by the editors. May be left out on submission.
%\paperdetails{
%  submitted=2016-08-10,
%  published=2016-10-11,
%  year=2016,
%  volume=1,
%  issue=1,
%  articlenumber=1,
%}
%%%%%%%%%%%%%%%%%%


\begin{document}

\title{Catala's alternative semantics}
% \subtitle{Preparing Articles for Programming}% optional
% \titlerunning{Preparing Articles for Programming} %optional, in case that the title is too long; the running title should fit into the top page column

\author[a]{Alain Delaët}
% \authorinfo{is a PhD student \email{alain.delaet--tixeuil@inria.fr}.}
\affiliation[a]{Prosecco \& Epicure, InRiA, France}

\keywords{programming journal, compiler verification, domain specific language} % please provide 1--5 keywords


%%%%%%%%%%%%%%%%%%%%%%%%%%%%%
% Please go to https://dl.acm.org/ccs/ccs.cfm and generate your Classification
% System [view CCS TeX Code] stanz and copy _all of it_ to this place.
%% From HERE

\begin{CCSXML}
  <ccs2012>
  <concept>
  <concept_id>10010405.10010455.10010458</concept_id>
  <concept_desc>Applied computing~Law</concept_desc>
  <concept_significance>300</concept_significance>
  </concept>
  <concept>
  <concept_id>10011007.10011006.10011039</concept_id>
  <concept_desc>Software and its engineering~Formal language definitions</concept_desc>
  <concept_significance>500</concept_significance>
  </concept>
  <concept>
  <concept_id>10011007.10011006.10011041</concept_id>
  <concept_desc>Software and its engineering~Compilers</concept_desc>
  <concept_significance>500</concept_significance>
  </concept>
  </ccs2012>
\end{CCSXML}

\ccsdesc[300]{Applied computing~Law}
\ccsdesc[500]{Software and its engineering~Formal language definitions}
\ccsdesc[500]{Software and its engineering~Compilers}

% To HERE
%%%%%%%%%%%%%%%%%%%%%%%

% \maketitle

% Please always include the abstract.
% The abstract MUST be written according to the directives stated in 
% http://programming-journal.org/submission/
% Failure to adhere to the abstract directives may result in the paper
% being returned to the authors.
% \begin{abstract}
%   TODO
% \end{abstract}


\section{Continuation semantics for Catala}

\begin{figure}

  \begin{enumerate}
    \item \label{fig:rulevar} $\leval \synvar x, \kappa, \sigma \reval \leadsto \lcont\kappa, \sigma, \sigma(x) \rcont$
    \item \label{fig:ruleapp} $\leval e_1\ e_2, \kappa, \sigma \reval \leadsto \leval e_1, (\square\ e_2) \cons \kappa, \sigma \reval $
    \item \label{fig:ruleclo} $\leval \lambda x. t, \kappa, \sigma \reval \leadsto \lcont\kappa, \sigma, Clo (x, t, \sigma) \rcont$ % 
    \item \label{fig:rulearg} $\lcont (\square\ e_2) \cons \kappa, \sigma, Clo (x, t_{cl}, \sigma_{cl}) \rcont \leadsto \leval e_2, (Clo(x, t_{cl}, \sigma_{cl})\ \square) \cons \kappa, \sigma \reval$
    \item  \label{fig:rulebeta} $\lcont(Clo(x, t_{cl}, \sigma_{cl})\ \square) \cons \kappa, \sigma, v \rcont \leadsto \leval t_{cl}, \kappa, (x\mapsto v) \cons\sigma_{cl} \reval$
    \item \label{fig:ruledefault} $\leval \synlangle es \synmid e_j \synjust e_c \synrangle, \kappa, \sigma \reval \leadsto \lcont (\mathtt{Def}(\synnone, es, e_j, e_c)) \cons \kappa, \sigma, \synempty \rcont$
    \item \label{fig:ruledefaultunpack} $\lcont \mathtt{Def}(o, e_h \cons es, e_j, e_c) \cons \kappa, \sigma, \synempty \rcont \leadsto \leval e_h, \mathtt{Def}(o, es, e_j, e_c) \cons \kappa, \sigma \reval$
    \item \label{fig:ruledefaultnone} $\lcont \mathtt{Def}(\synnone, e_h \cons es, e_j, e_c) \cons \kappa, \sigma, v \rcont \leadsto \leval e_h, \mathtt{Def}(\synsome(v), es, e_j, e_c) \cons \kappa, \sigma \reval$
    \item  \label{fig:ruledefaultconflict} $\lcont \mathtt{Def}(\synsome(v), es, e_j, e_c) \cons \kappa, \sigma, v' \rcont \leadsto \lcont \mathtt{Def}(\synsome(v), es, e_j, e_c)  \cons \kappa, \sigma, \synconflict \rcont$
    \item \label{fig:ruledefaultvalue} $\lcont \mathtt{Def}(\synsome(v), [], e_j, e_c) \cons \kappa, \sigma, \synempty \rcont \leadsto \lcont \kappa, \sigma, v \rcont$
    \item \label{fig:ruledefaultnonefinal} $\lcont \mathtt{Def}(\synnone, [], e_j, e_c) \cons \kappa, \sigma, v \rcont \leadsto \lcont \kappa, \sigma, v \rcont$ \hfill $v \neq \synempty, \synconflict$
    \item \label{fig:ruledefaultbase} $\lcont \mathtt{Def}(\synnone, [], e_j, e_c) \cons \kappa, \sigma, \synempty \rcont \leadsto \leval e_j, \synlanglemid \square \synjust e_c \synrangle \cons \kappa, \sigma \reval$
    \item \label{fig:ruledefaultbasetrue} $\lcont \synlanglemid \square \synjust e_c \synrangle \cons \kappa, \sigma, \syntrue \rcont \leadsto \leval e_c, \kappa, \sigma \reval$
    \item \label{fig:ruledefaultbasefalse} $\lcont \synlanglemid \square \synjust e_c \synrangle \cons \kappa, \sigma, \synfalse \rcont \leadsto \lcont \kappa, \sigma, \synempty \rcont$
          \item\label{fig:ruleempty} $\lcont \phi \cons \kappa, \sigma, \synempty \rcont \leadsto \lcont \kappa, \sigma, \synempty \rcont$ \hfill $\forall o\ es\ e_j\ e_c,\ \phi \neq \mathtt{Def}(o, es, e_j, ec)$
          \item\label{fig:ruleconflict} $\lcont \phi \cons \kappa, \sigma, \synconflict \rcont \leadsto \lcont \kappa, \sigma, \synconflict \rcont$
  \end{enumerate}
  \caption{\label[figure]{fig_contsem} Continuation style semantics for $\lambda_d$.}
\end{figure}


TODO: rewrite as it is copied/paste from Adam \& Alan's paper

Continuations semantics is an alternative way in describing semantics. It provides fined grained informations as the small steps, but permit to have structural induction in a easier way than usual. To define such a semantics, one need two different modes of computations: The evaluation mode, written as $\leval e, \kappa, \sigma\reval$, evaluates $e$ with a continuation $\kappa$ and an environment $\sigma$. Similarly, the continuation mode, denoted by $\lcont \kappa, \sigma, r\rcont$ applies the continuation $\kappa$ to the computed result $r$. A result is either a pure value, an empty error $\synempty$ or a conflict error $\synconflict$. Continuations are terms with holes, or closures, or Default closures.

\Cref{fig_contsem} introduces all the rules for the continuation semantics for the $\lambda_d$ language. Rules \crefrange{fig:rulevar}{fig:rulebeta} are classical $\lambda$-calculus rules taken from \cite{khayam_practical}. Rules \crefrange{fig:ruledefault}{fig:ruleconflict} are specific to the default term.




\section{Technical lemmas}

To prove invariance of the computation if we swap differents exceptions in a default term, we need to be able to use have general rules of computation. This requires technical lemmas about inversion of the stack.

The first one is the following lemma, stating we can append any stack by an other stack. This is fairly useful to show theorems working on empty stacks, and then lifting them to arbitrary stacks.

\begin{lemma}
If $s_1 \leadsto s_2$, then $s_1' \leadsto s_2'$ where $s_i' = \mathtt{with\_stack} s_i (\mathtt{stack}(s_i) ++ \kappa)$.
\begin{lemma}
\begin{proof}
  The proof is straight forward by taking a look to all the cases.
\end{proof}

The inverse lemma does not hold in whole generality. Indeed, if the top of the stack may be modified at any point rules of kind bla. Hence, we only need to prove the following lemma stating that if the bottom of the stack is unchanged, then we can apply the rule without the bottom of the stack. In particular, this means

\begin{lemma}
  If $s_1 \leadsto s_2$

\end{lemma}



\begin{lemma}
  Let $x, k_1, \sigma_1, \sigma_2, v_3$ such that $\leval x, [k_1], \sigma_1 \reval \leadsto^* \lcont [], \sigma_3, v_3 \rcont$. Then, there exists $v_2, \sigma_2$ such that $\leval x, [], \sigma_1 \reval \leadsto^* \lcont [], \sigma_2, v_2 \rcont$.
\end{lemma}

\begin{proof}
  Since $\mathtt{List.length}(\mathtt{stack} \leval x, [k], \sigma_1 \reval) = 1$ and $\mathtt{List.length} (\lcont [], \sigma_3, v_3 \rcont) = 0$, there is an $s_2 \leadsto s_2'$ first in the sequence $\leval x, [k_1], \sigma_1 \reval \leadsto^* \lcont [], \sigma_3, v_3 \rcont$ such that $\mathtt{List.length}(\mathtt{stack}(s_2)) \geq 1$ and $\mathtt{List.length}(\mathtt{stack}(s_2')) = 0$. Since stack size vary by one at most and is deacreasing, we know $s_2$ is in the form $\lcont [k_2], \sigma_2, v_2 \rcont$. Finally, since $s_2 \leadsto s_2'$ is the first, the stack size between $\leval x, [k_1], \sigma_1 \reval$ and $\lcont [k_2], \sigma_2, v_2 \rcont$ is always greater than one. Hence $k_1 = k_2$ and $\leval x, [], \sigma_1 \reval \leadsto^* \lcont [], \sigma_2, v_2 \rcont$. This concludes the proof.
\end{proof}


\printbibliography
\end{document}

% Local Variables:
% TeX-engine: luatex
% End:
